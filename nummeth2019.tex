\documentclass[12pt,a4paper,openany]{book}
\usepackage[utf8]{inputenc} % кодировка
\usepackage[english, russian]{babel} % Русские и английские переносы
\usepackage{cite}              % для корректного оформления литературы
\usepackage{enumitem}
\usepackage{amsmath}
\usepackage{amsfonts}
\usepackage{amssymb}
\usepackage{graphicx}
\usepackage{listings}
\usepackage{color}
\usepackage{multirow}
\usepackage[breakwords]{truncate}
\usepackage{units}
\usepackage{titletoc}% http://ctan.org/pkg/titletoc
\usepackage{lastpage}
\usepackage[ruled,vlined]{algorithm2e}
\usepackage{makeidx}

%\usepackage[]{caption}

\usepackage{geometry}
\geometry{papersize={21cm, 29.7cm}}
\geometry{left=2cm}
\geometry{right=2cm}
\geometry{top=3cm}
\geometry{bottom=3cm}

\lstdefinelanguage{shell}{
	morekeywords={cd,ls,mkdir,cp,touch, make},
	otherkeywords={=>,<-,<\%,<:,>:,@},
	sensitive=true,
	morecomment=[l]{\#},
	morestring=[b]",
	morestring=[b]',
	morestring=[b]"""
}


\usepackage[most]{tcolorbox}

\definecolor{dkgreen}{rgb}{0,0.6,0}
\definecolor{gray}{rgb}{0.3,0.3,0.3}
\definecolor{lightgray}{rgb}{0.97,0.97,0.97}
\definecolor{mauve}{rgb}{0.58,0,0.82}

\tcbset{
    frame code={}
    left title,
    left=0pt,
    right=0pt,
    top=0pt,
    bottom=0pt,
    colback=gray!10,
    colframe=red!75,
    width=\dimexpr\textwidth\relax,
    enlarge left by=0mm,
    boxsep=5pt,
    %arc=0pt,outer arc=0pt,
    leftrule=0cm,
    }

\lstset{ %
	backgroundcolor=\color[rgb]{0.98,0.98,0.98},
	frame=lines,
	%captionpos=b,
	language=C++,
	aboveskip=3mm,
	belowskip=7mm,
	showstringspaces=false,
	columns=flexible,
	basicstyle={\small\ttfamily},
	numbers=none,
	numberstyle=\tiny\color{gray},
	keywordstyle=\bf\color{black},
	commentstyle=\color{dkgreen},
	stringstyle=\sf\color{gray},
	breaklines=true,
	texcl=true,
	%breakatwhitespace=true,
	tabsize=3,
	%frame=none,
extendedchars=true,
literate={Ö}{{\"O}}1
{Ä}{{\"A}}1
{Ü}{{\"U}}1
{ß}{{\ss}}1
{ü}{{\"u}}1
{ä}{{\"a}}1
{ö}{{\"o}}1
{~}{{\textasciitilde}}1
{а}{{\selectfont\char224}}1
{б}{{\selectfont\char225}}1
{в}{{\selectfont\char226}}1
{г}{{\selectfont\char227}}1
{д}{{\selectfont\char228}}1
{е}{{\selectfont\char229}}1
{ё}{{\"e}}1
{ж}{{\selectfont\char230}}1
{з}{{\selectfont\char231}}1
{и}{{\selectfont\char232}}1
{й}{{\selectfont\char233}}1
{к}{{\selectfont\char234}}1
{л}{{\selectfont\char235}}1
{м}{{\selectfont\char236}}1
{н}{{\selectfont\char237}}1
{о}{{\selectfont\char238}}1
{п}{{\selectfont\char239}}1
{р}{{\selectfont\char240}}1
{с}{{\selectfont\char241}}1
{т}{{\selectfont\char242}}1
{у}{{\selectfont\char243}}1
{ф}{{\selectfont\char244}}1
{х}{{\selectfont\char245}}1
{ц}{{\selectfont\char246}}1
{ч}{{\selectfont\char247}}1
{ш}{{\selectfont\char248}}1
{щ}{{\selectfont\char249}}1
{ъ}{{\selectfont\char250}}1
{ы}{{\selectfont\char251}}1
{ь}{{\selectfont\char252}}1
{э}{{\selectfont\char253}}1
{ю}{{\selectfont\char254}}1
{я}{{\selectfont\char255}}1
{А}{{\selectfont\char192}}1
{Б}{{\selectfont\char193}}1
{В}{{\selectfont\char194}}1
{Г}{{\selectfont\char195}}1
{Д}{{\selectfont\char196}}1
{Е}{{\selectfont\char197}}1
{Ё}{{\"E}}1
{Ж}{{\selectfont\char198}}1
{З}{{\selectfont\char199}}1
{И}{{\selectfont\char200}}1
{Й}{{\selectfont\char201}}1
{К}{{\selectfont\char202}}1
{Л}{{\selectfont\char203}}1
{М}{{\selectfont\char204}}1
{Н}{{\selectfont\char205}}1
{О}{{\selectfont\char206}}1
{П}{{\selectfont\char207}}1
{Р}{{\selectfont\char208}}1
{С}{{\selectfont\char209}}1
{Т}{{\selectfont\char210}}1
{У}{{\selectfont\char211}}1
{Ф}{{\selectfont\char212}}1
{Х}{{\selectfont\char213}}1
{Ц}{{\selectfont\char214}}1
{Ч}{{\selectfont\char215}}1
{Ш}{{\selectfont\char216}}1
{Щ}{{\selectfont\char217}}1
{Ъ}{{\selectfont\char218}}1
{Ы}{{\selectfont\char219}}1
{Ь}{{\selectfont\char220}}1
{Э}{{\selectfont\char221}}1
{Ю}{{\selectfont\char222}}1
{Я}{{\selectfont\char223}}1
{і}{{\selectfont\char105}}1
{ї}{{\selectfont\char168}}1
{є}{{\selectfont\char185}}1
{ґ}{{\selectfont\char160}}1
{І}{{\selectfont\char73}}1
{Ї}{{\selectfont\char136}}1
{Є}{{\selectfont\char153}}1
{Ґ}{{\selectfont\char128}}1
}


\makeatletter%ставим, чтобы знак @ воспринимался как буква, а не как команда

\renewcommand{\@oddhead}{\hbox to 135mm{\hrulefill\raisebox{2.2mm}{\underline{\strut{\small\bfseries\slshape\truncate{170mm}{\rightmark}}}}}}%верхний колонтитул для нечетных страниц

\renewcommand{\@oddfoot}{\hbox to 170mm{\hfil\thepage\hfil}}%%нижний колонтитул для нечетных страниц

%\renewcommand{\@evenhead}{\hbox to 135mm{\raisebox{2.2mm}{\underline{\strut{\small\bfseries\slshape \thechapter}}}\hrulefill}}%верхний колонтитул для четных страниц.

\renewcommand{\@evenhead}{\hbox to 135mm{\raisebox{2.2mm}{\underline{\strut{\small\bfseries\slshape\truncate{170mm}{\leftmark}}}}\hrulefill}}%верхний колонтитул для четных страниц.

\renewcommand{\@evenfoot}{\hbox to 170mm{\hfil\thepage\hfil}}%нижний колонтитул для четных страниц



\renewcommand{\@biblabel}[1]{#1.} % Заменяем библиографию с квадратных скобок на точку:


\makeatother% возвращаем знаку @ командные свойства

\titlecontents{chapter}% <section-type>
  [0pt]% <left>
  {\addvspace{1em}}% <above-code>
  {\bfseries\chaptername\ \thecontentslabel\quad}% <numbered-entry-format>
  {\bfseries\hspace{-0.6cm}\thecontentslabel\quad}% <numberless-entry-format>
  {\hfill\contentspage}% <filler-page-format>


\renewcommand{\thechapter}{\arabic{chapter}.}
\renewcommand{\thesection}{\arabic{chapter}.\arabic{section}.}
\renewcommand{\thesubsection}{\arabic{chapter}.\arabic{section}.\arabic{subsection}.}
\renewcommand{\thefigure}{\arabic{chapter}.\arabic{figure}}
\renewcommand{\thetable}{\arabic{chapter}.\arabic{table}}
\renewcommand{\theequation}{\arabic{chapter}.\arabic{equation}}

\renewcommand{\chaptermark}[1]{\markboth{\emph{\chaptername\ \thechapter\ #1}}{}}

\renewcommand{\sectionmark}[1]{\markright{\emph{\thesection\ #1}}}

\makeatletter
\long\def\@makecaption#1#2{%       
	\vskip\abovecaptionskip\footnotesize
	\sbox\@tempboxa{#2}         % place contents of #2 into a scratch TeX box
	\ifdim \wd\@tempboxa = 0pt  % test if scratch box has zero width
	\centering #1 \par       % if yes, typeset only #1 (the float's name and number)
	\else                       % if no, proceed with default definition
	\sbox\@tempboxa{#1. #2}%
	\ifdim \wd\@tempboxa >\hsize
	#1. #2\par
	\else
	\global \@minipagefalse
	\hb@xt@\hsize{\hfil\box\@tempboxa\hfil}%
	\fi
	\fi
	\vskip\belowcaptionskip}

%\renewcommand{\@algocf@capt@plain}{above}% formerly {bottom}


\makeatother

%\author{Жалнин Р.В., Масягин В.Ф., Панюшкина Е.Н., Пескова Е.Е.}
%\title{Использование технологии CUDA для решения задач математической физики}

\newcommand{\zhrvAuthors}{Пушкин А.С.}
\newcommand{\zhrvTitle}{Капитанская дочка}
\newcommand{\zhrvISBN}{978-5-901661-41-3}

\newenvironment{zhgit}{\vskip 10pt\begin{tcolorbox}\begin{tabular}{ll}\noindent\parbox[c]{20pt}{\includegraphics*[scale=0.06]{img/Git-Icon-Black.png}} & \tt\small
}
{
\end{tabular}\end{tcolorbox}\vskip 10pt}

\newcommand{\pard}[2]{\frac{\displaystyle\partial #1}{\displaystyle\partial #2}}
\newcommand{\pardtwo}[2]{\frac{\displaystyle\partial^2 #1}{\displaystyle\partial #2^2}}


\makeindex

\begin{document}
\renewcommand{\contentsname}{\vspace{0mm}Содержание}
\renewcommand{\thelstlisting}{\thesection\arabic{lstlisting}}

\fontsize{14}{16pt}\selectfont
	\begin{titlepage}
	\centering
	{\scshape Министерство образования и науки Российской Федерации\\ 
		Федеральное государственное бюджетное образовательное\\
		учреждение высшего образования <<Национальный\\
		исследовательский Мордовский государственный\\
		университет имени Н.~П.~Огарёва>>\par}
	\vspace{5cm}
	%
	\vfill
	{\huge\bfseries\LARGE \zhrvTitle\par}
	\vspace{5cm}
	{\scshape\Large А.~С.~Пушкин\par}
	\vspace{2cm}
	\vfill
	
	% Bottom of the page
	{\large САРАНСК\\2022\par}
\end{titlepage}

\newpage
\thispagestyle{empty}
УДК 519.68

ББК 22.193

\hspace{1.25cm}Ж24
\ \\
	\vspace{1cm}
	\begin{center}
	\parbox{160mm}{
		\parbox{43mm}{
			А~в~т~о~р~ы:
		}
		\parbox{117mm}{
			И.~И.~Иванов, П.~П.~Петров
		}
	}
	\end{center}
	\vspace{0.3cm}
	\begin{center}
		\parbox{160mm}{
			\parbox{43mm}{
				Р~е~ц~е~н~з~е~н~т:\\ \\ \\
			}
			\parbox{117mm}{
				доцент кафедры <<Высшая и прикладная математика>> Пензенского государственного университета, кандидат физико-математических наук, доцент С.~С.~Сидоров
			}
		}
	\end{center}
	\vspace{0.3cm}
	\parbox{160mm}{%
		\parbox{13mm}{\vspace{-1mm}\bfseries Ж24}
		\parbox{147mm}{\bfseries \hspace{5mm} \zhrvTitle: роман / А.~С.~Пушкин. -- Саранск~: Изд-во СВМО, 2019. -- \pageref{LastPage}~с.}
	}\par
	\vspace{0.6cm}
	\parbox{160mm}{%
		\parbox{7mm}{\ }
		\parbox{153mm}{
			ISBN \zhrvISBN
			
			\vspace{0.6cm}
			%{\it Печатается в авторской редакции.\par}
			{\small Пособие 
			содержит изложение основ разработки программного кода, реализующего вычислительные алгоритмы для решения задач математической физики. Дается описание современных средств генерации сеток, описание форматов файлов, используемых для представления результатов расчетов. Приводится пошаговое описание процесса создания кода для численного решения краевой задачи для  уравнения теплопроводности и задачи о течении идеального газа с использованием метода конечных объемов и метода Галеркина с разрывными базисными функциями.
			
			
			Предназначено для студентов-бакалавров и магистрантов, начинающих свое знакомство с современными численными методами прикладной математики.
			\par}
		}
	}\par
	\vspace{2.6cm}
	\parbox{160mm}{%
		\parbox{7mm}{\ }
		\parbox{153mm}{
			Публикуется на основании Устава Средневолжского математического общества (пп.~2.4 и 2.5) и по решению редакционного отдела СВМО
		}
	}
	
	%{\it Печатается в авторской редакции.\par}
	%\vspace{0.6cm}
	%\vspace{5cm}
	\vfill
	%\hspace{6.5cm}
	\parbox{16.5cm}{
		\parbox{7.5cm}{
			\vspace{-0.5cm}\hspace{-0.6cm}ISBN \zhrvISBN\\
		}
		\parbox{0.7cm}{
			\textcopyright \\
			\textcopyright \\
		}
		\parbox{9cm}{
			А.~С.~Пушкин,~2022 \\
			Средневолжское \\математи\-ческое общество,~2022
		}	
	}

	\vfill	


	\tableofcontents
	\include{chapter1}
	\include{chapter2}
	\include{chapter3}
	\include{chapter4}
	\include{chapter5}
	\include{chapter6}
	\include{chapter7}
	\include{chapter8}
	\include{chapter9}
	\include{chapter10}
	\include{chapter11}
	\include{chapter12}
	\include{chapter13}
	\include{chapter14}
	\include{chapter15}
		\newpage
	\thispagestyle{empty}
	\printindex
	
\newpage
\thispagestyle{empty}

\begin{center}
\ \\
	\vspace{6cm}

	Учебное издание
	
	\vspace{1cm}
	\textbf{ИВАНОВ Иван Иванович}\\

	
	\vspace{1cm}
	\textbf{\LARGE\zhrvTitle}
	
	\vspace{0.5cm}
	\textbf{Учебное пособие}
	
	\vspace{2cm}
	{\it Печатается в авторской редакции\\ в соответствии с представленным оригинал-макетом}
	
	\vspace{1.5cm}
	
		
	\vspace{1cm}
	Подписано в печать 14.11.19.\\ 
	Формат $\displaystyle 60\times84\ \nicefrac{1}{16}$.
	%Усл. печ. л. ???4,42. 
	Тираж 100.\\
	\vspace{0.2cm}
	Издательство Средневолжского математического общества\\
	Отпечатано в ООО <<Типография <<Полиграф>>\\
	430005, г. Саранск, ул. Рабочая, 155

\end{center}	
	

\end{document}