\begin{titlepage}
	\centering
	{\scshape Министерство образования и науки Российской Федерации\\ 
		Федеральное государственное бюджетное образовательное\\
		учреждение высшего образования <<Национальный\\
		исследовательский Мордовский государственный\\
		университет имени Н.~П.~Огарёва>>\par}
	\vspace{5cm}
	%
	\vfill
	{\huge\bfseries\LARGE \zhrvTitle\par}
	\vspace{5cm}
	{\scshape\Large А.~С.~Пушкин\par}
	\vspace{2cm}
	\vfill
	
	% Bottom of the page
	{\large САРАНСК\\2022\par}
\end{titlepage}

\newpage
\thispagestyle{empty}
УДК 519.68

ББК 22.193

\hspace{1.25cm}Ж24
\ \\
	\vspace{1cm}
	\begin{center}
	\parbox{160mm}{
		\parbox{43mm}{
			А~в~т~о~р~ы:
		}
		\parbox{117mm}{
			И.~И.~Иванов, П.~П.~Петров
		}
	}
	\end{center}
	\vspace{0.3cm}
	\begin{center}
		\parbox{160mm}{
			\parbox{43mm}{
				Р~е~ц~е~н~з~е~н~т:\\ \\ \\
			}
			\parbox{117mm}{
				доцент кафедры <<Высшая и прикладная математика>> Пензенского государственного университета, кандидат физико-математических наук, доцент С.~С.~Сидоров
			}
		}
	\end{center}
	\vspace{0.3cm}
	\parbox{160mm}{%
		\parbox{13mm}{\vspace{-1mm}\bfseries Ж24}
		\parbox{147mm}{\bfseries \hspace{5mm} \zhrvTitle: роман / А.~С.~Пушкин. -- Саранск~: Изд-во СВМО, 2019. -- \pageref{LastPage}~с.}
	}\par
	\vspace{0.6cm}
	\parbox{160mm}{%
		\parbox{7mm}{\ }
		\parbox{153mm}{
			ISBN \zhrvISBN
			
			\vspace{0.6cm}
			%{\it Печатается в авторской редакции.\par}
			{\small Пособие 
			содержит изложение основ разработки программного кода, реализующего вычислительные алгоритмы для решения задач математической физики. Дается описание современных средств генерации сеток, описание форматов файлов, используемых для представления результатов расчетов. Приводится пошаговое описание процесса создания кода для численного решения краевой задачи для  уравнения теплопроводности и задачи о течении идеального газа с использованием метода конечных объемов и метода Галеркина с разрывными базисными функциями.
			
			
			Предназначено для студентов-бакалавров и магистрантов, начинающих свое знакомство с современными численными методами прикладной математики.
			\par}
		}
	}\par
	\vspace{2.6cm}
	\parbox{160mm}{%
		\parbox{7mm}{\ }
		\parbox{153mm}{
			Публикуется на основании Устава Средневолжского математического общества (пп.~2.4 и 2.5) и по решению редакционного отдела СВМО
		}
	}
	
	%{\it Печатается в авторской редакции.\par}
	%\vspace{0.6cm}
	%\vspace{5cm}
	\vfill
	%\hspace{6.5cm}
	\parbox{16.5cm}{
		\parbox{7.5cm}{
			\vspace{-0.5cm}\hspace{-0.6cm}ISBN \zhrvISBN\\
		}
		\parbox{0.7cm}{
			\textcopyright \\
			\textcopyright \\
		}
		\parbox{9cm}{
			А.~С.~Пушкин,~2022 \\
			Средневолжское \\математи\-ческое общество,~2022
		}	
	}

	\vfill	

